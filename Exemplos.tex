\chapter{Introdução}


Este Manual traz os procedimentos para a elaboração e submissão de uma CONSULTA DE ACESSO de uma Ligação Nova na plataforma de {\color{blue} \textbf{Projetos Particulares da CPFL}}. Ele faz parte de um grupo de outros manuais elaborados pela {\color{blue} \textbf{DINÂMICA}}, a fim de padronizar os processos internos de engenharia relacionados à projetos e homologação de usinas solares fotovoltaicas de potência superior a 75 kW, enquadadas como \textbf{``minigeração distribuída''} na concessionária {\color{CPFL}}. 

\section{Consulta de Acesso}

A \textbf{Consulta de Acesso} trata-se de uma pedido formal protocolado na concessionária informando que deseja-se fazer uma conexão em um determinado local com uma determinada carga, seja ela somente de geração ou não.

A Consulta pode ser feita para locais onde ainda não exista unidades consumidoras ou em locais onde já exista uma unidade consumidora com um padrão de entrada. Dessa forma pode ser verificado se é viável o acesso que pretende-se fazer com a implantação da usina solar fotovoltaica no local.

\section{Informação de Acesso}

A \textbf{Informação de Acesso} é um documento fornecido pela CPFL, em retorno e como resposta à \textbf{Solicitação de Acesso}. A Informação de acesso traz uma análise completa a respeito da possível conexão da usina fotovoltaico na rede elétrica, indicando se é necessária uma obra na rede elétrica da concessionária. Caso sejam necessárias obras na rede, a Informação de Acesso traz uma estimativa dos valores em dinheiro que serão gastos, tanto pelo cliente, quanto pela própria concessionária.

A Figura \ref{fig:CapaInfoAcesso} traz a capa de uma Informação de Acesso e a Figura \ref{fig:DadosCC} traz um exemplo de Dados de Curto-Circuito, ambos fornecidos pela CPFL como resposta à uma Consulta de Acesso aprovada.


%%%%%%%%%%%%%%%%%%%%%%%%%%%%%%%%%%%%%%%%%%%%%%%%%%%%%%%%%%%%%%%%
%%LDN FIGURA dos documentos recebidos deposis Consulta de Acesso
%%%%%%%%%%%%%%%%%%%%%%%%%%%%%%%%%%%%%%%%%%%%%%%%%%%%%%%%%%%%%%%%
    \begin{figure}[H]
        \centering
        \subfigure[\label{fig:CapaInfoAcesso}Informação de Acesso]
        {\includegraphics[width=0.45\textwidth]{Figuras/Intro_ex_info_acesso.png}}
        \qquad
        \subfigure[\label{fig:DadosCC}Dados de Curto-Circuito]
        {\includegraphics[width=0.45\textwidth]{Figuras/Intro_ex_dados_cc.png}}
        \caption{\label{fig:RespostasConsulta}Exemplos dos documentos retornados pela CPFL\\em resposta à uma Consulta de Acesso.}
    \end{figure}

\section{Prazos da Consulta e da Informação de Acesso}

Em suma, de acordo com o Módulo 3 do Procedimentos de Distribuição de Energia Elétrica no Sistema Elétrico Nacional (PRODIST) {\color{blue} \cite{Modulo3Prodist}}, a distribuidora acessada deve apresentar a informação de acesso ao acessante em até 60 (sessenta) dias, contados a partir da data de recebimento da Consulta de Acesso.

Assim como ilustrado na Figura \ref{fig:fluxoConsInfo}, depois do prazo máximo estabelecido pela ANEEL, a concessioária pode aprovar ou reprovar a viabilidade. Quando a viabilidade é aprovada, a CPFL envia de volta a \textbf{Informação de Acesso} junto com os \textbf{Dados de Curto-Circuito}, caso contrário esses dados de curto-circuito não são recebidos. 

%%%%%%%%%%%%%%%%%%%%%%%%%%%%%%%%%%%%%%%%%%%%%%
%%LDN FIGURA de um suporte de fixação genérico
%%%%%%%%%%%%%%%%%%%%%%%%%%%%%%%%%%%%%%%%%%%%%%
    \begin{figure} [H]
        \centering
        \includegraphics[width=0.75\textwidth]{Figuras/Intro_fluxograma_Cons_Info.png}
        \caption{Fluxograma com prazo do recebimento da Informação de Acesso}
        \label{fig:fluxoConsInfo}
    \end{figure}

\section{Documentos necessários para a Consulta}

A Consulta de Acesso deve ser protocolada na plataforma de Projetos Particulares da CPFL (\href{https://projetosparticulares.cpfl.com.br/}{\color{blue}https://projetosparticulares.cpfl.com.br/}) pelo engenheiro responsável e lá são anexados uma série da documentos que são montados pelo projetista. Para a Consulta de Acesso de Nova Ligação, ou seja, quando não existe uma unidade consumidora (UC), são exigidos os seguintes documentos:

De acordo com experiências práticas e com orientações da CPFL em \cite{CartilhaCpfl}, segue a lista de documentos necessários para submeter uma Consulta de Acesso:\\

\hspace{0.2cm} \textbf{ $\bullet$ \hspace{0.1cm} Memorial Descritivo Simplificado;}\\[-0.5cm]

\hspace{0.2cm} \textbf{ $\bullet$ \hspace{0.1cm} Diagrama Unifilar - Localização;}\\[-0.5cm]

\hspace{0.2cm} \textbf{ $\bullet$ \hspace{0.1cm} Anexo VII - Formulário (GED4732);}\\[-0.5cm]

\hspace{0.2cm} \textbf{ $\bullet$ \hspace{0.1cm} Anexo F (GED15303);}\\[-0.5cm]

\hspace{0.2cm} \textbf{ $\bullet$ \hspace{0.1cm} Carta de Apresentação do Projeto (Anexo I da GED4732);}\\[-0.5cm]

\hspace{0.2cm} \textbf{ $\bullet$ \hspace{0.1cm} Datasheet dos Módulos;}\\[-0.5cm]

\hspace{0.2cm} \textbf{ $\bullet$ \hspace{0.1cm} Datasheet dos Inversores;}\\[-0.5cm]

\hspace{0.2cm} \textbf{ $\bullet$ \hspace{0.1cm} Certificado de Registro Profissional e Anotações;}\\[-0.5cm]

\hspace{0.2cm} \textbf{ $\bullet$ \hspace{0.1cm} CNH do Eng. Responsável;}\\

Nos próximos capítulos serão descritos com mais detalhes as caracteríssticas dos documentos que deverão ser elaborados pelo projetista.
